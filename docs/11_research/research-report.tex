% research-report.tex: 调研报告
% Copyright (C) 2022 吴骏东, 张子辰, 蓝俊玮, 郭耸霄 and 陈建绿
% All rights reserved.
\documentclass[UTF8,fontset=none,linespread=1.15]{ctexart}
\ctexset
{
    section/format={\Large\sffamily\bfseries},
    subsection/format+={\sffamily},
    subsubsection/format+={\itshape}
}
\setCJKmainfont[ItalicFont={KaiTi},BoldItalicFont={KaiTi},
BoldItalicFeatures={FakeBold=3}]{Noto Serif CJK SC}
\setCJKsansfont[AutoFakeSlant]{Noto Sans CJK SC DemiLight}
\setCJKmonofont[AutoFakeBold=3,AutoFakeSlant]{FangSong}
\setmainfont{cmun}[Extension=.otf,UprightFont=*rm,
ItalicFont=*ti,BoldFont=*bx,BoldItalicFont=*bi]
\setsansfont{cmun}[Extension=.otf,UprightFont=*ss,
ItalicFont=*si,BoldFont=*sx,BoldItalicFont=*so]
\setmonofont{cmun}[Extension=.otf,UprightFont=*btl,
ItalicFont=*bto,BoldFont=*tb,BoldItalicFont=*tx]
%Computer Modern Unicode 的\textasciitilde和\~{}的高度相同,所以用\tildechar表示居中的波浪线~
\newcommand{\tildechar}{\raisebox{-0.35em}{\textasciitilde}}
\usepackage[a4paper,hmargin=1.2in,vmargin=1in]{geometry}
\usepackage{graphicx,tikz,float,subfig,multicol,makecell,multirow,longtable}
\usepackage[normalem]{ulem}
\usepackage{CJKfntef}
\usepackage[perpage]{footmisc}

%目录, 参考了OSH-2021/x-sBPF
\usepackage{titletoc}
\titlecontents{section}[2em]{\addvspace{1.3mm}\bfseries}{%
\contentslabel{2.0em}}{}{\titlerule*[5pt]{$\cdot$}\contentspage}
\titlecontents{subsection}[4.2em]{}{\contentslabel{2.5em}}{}{%
\titlerule*[5pt]{$\cdot$}\contentspage}
\titlecontents{subsubsection}[7.2em]{}{\contentslabel{3.3em}}{}{%
\titlerule*[5pt]{$\cdot$}\contentspage}

%代码环境
\usepackage{listings}
\lstset{basicstyle={\normalfont\ttfamily},breaklines,tabsize=4}

\usepackage{enumitem}
\setlistdepth{5}
\renewlist{enumerate}{enumerate}{5}
\setlist{itemsep=0pt,partopsep=0pt,parsep=0pt,topsep=0pt}
\setlist[enumerate,1]{label=\arabic*.}
\setlist[enumerate,2]{label=(\arabic*)}
\setlist[enumerate,3]{label=\textcircled{\arabic*}}
\setlist[enumerate,4]{label=(\textit{\roman*})}
\setlist[enumerate,5]{label=\textit{\alph*})}

\usepackage[colorlinks,unicode,pdfstartview={FitH}]{hyperref}
\hypersetup
{
  pdftitle={2022春 操作系统原理与设计(H) x-runikraft小组 调研报告},
  pdfauthor={吴骏东; 张子辰; 蓝俊玮; 郭耸霄; 陈建绿}
}

%带圈数字,它必须在hyperref之后载入
\usepackage{xunicode-addon}
\makeatletter
\xeCJKDeclareCharClass{Default}{"24EA, "2460->"2473, "3251->"32BF,"24B6->"24E9,"2160->"217F}
\newfontfamily\EnclosedNumbers{Noto Serif CJK SC}
\AtBeginUTFCommand[\textcircled]{\begingroup\EnclosedNumbers}
\AtEndUTFCommand[\textcircled]{\endgroup}
\makeatother

\makeatletter
\let\textcircled@old\textcircled
\protected\def\textcircled#1{%
	\expandafter\textcircled@old\expandafter{\expanded{#1}}}
\makeatother
\makeatletter
\renewcommand\@makefntext[1]{%
	\setlength\parindent{0.75\ccwd}\selectfont
	\@thefnmark\ #1}
\makeatother
\renewcommand*\thefootnote{\textcircled{\arabic{footnote}}}
\renewcommand{\lstlistingname}{代码}

%上标+引用
\let\nosupcite\cite
\renewcommand*{\cite}[1]{\textsuperscript{\nosupcite{#1}}}

%章节作者
\newcommand{\sectionauthor}[1]{%
\vspace*{-5ex}
\noindent\textrm{\hfill\textit{by #1}}
\vspace*{3ex}\par}
%\renewcommand{\today}{2022年3月27日}

\begin{document}
\sffamily %为方便屏幕阅读,文档主要使用无衬线字体
\title{\bfseries x-runikraft小组\quad 调研报告}
\author{吴骏东\and 张子辰\and 蓝俊玮\and 郭耸霄\and 陈建绿}
\date{\today}
\maketitle

\tableofcontents

\section{项目简介}
\textbf{本项目参考 Unikraft 的设计,用 Rust 语言实现模块化的 unikernel
在保持 Unikraft 的 POSIX 兼容性、可定制性的基础上,用 Rust 语言增强
内核的安全性。}

Unikernel是专一用途的、单地址空间的轻量操作系统。Unikernels在虚拟机
上运行时,能够提供比传统的容器更短的启动时间、更高的运行效率和更强的
隔离性,因此unikernels通常被用在云计算领域。\cite{bib:unikernel}
然而,为了追求轻量性,
unikernels裁剪了传统的操作系统的众多组件,因此unikernels无法提供许多
常用的库的应用程序接口,所以为了将现有的程序移植到某个unikernel平台,开发者
不得不根据该unikernel的API重构程序。\cite{bib:unikraft}
此外,为了轻量、快速,unikernels
删去的许多基本的并且不会影响性能的安全措施,这导致unikernels相比容器
更容易受到用户程序的安全漏洞的影响。\cite{bib:unikernel-secuirty}

Unikraft是一个充分考虑了兼容性和安全性的unikernel,它将系统分割成
若干相对独立的模块,各个模块可以独立安装和更新,就像传统操作系统上的动态库。
在创建系统镜像时,Unikraft提供的编译系统能够编译用户需要使用的模块,并将
它们与用户代码一起连接成可引导镜像。

我们小组计划仿照Unikraft的架构,用Rust语言编写能在RISC-V架构+ KVM平台上
运行的unikernel——Runikraft。Runikraft的核心代码使用Rust编写,但
允许用户代码使用任何语言编写——只要它能够被编译成入口为\texttt{main}的
目标代码。Runikraft强调构建系统镜像的简洁,用户只需要
修改现有的项目的编译参数就可以构建基于Runikraft的系统镜像,而不必使用
专用的工具链,更不需要重构代码。Runikraft是POSIX兼容的,所以它将支持
内存管理、进程调度,甚至磁盘管理和进程通信。不过,这些功能都是可选的且可
拓展的,如果用户不需要某项功能,他可以不将相关模块打包进系统镜像中,
如果用户能够提供某些功能的更好实现,他可以用自己的实现替换原有的模块,
甚至POSIX兼容层本身也是可选的,如果用户愿意为了效率重构代码,他也可以
直接用Runikraft的专用API。
Runikraft可以支持多进程,因为我们认为,将若干密切管理的程序打包到一个
镜像会提高效率。
\section{项目背景}
\subsection{Rust语言}\sectionauthor{郭耸霄}

这一部分主要阐释我们为什么选择Rust语言作为开发语言。Rust 是由 Mozilla 研究室
主导开发的一门现代系统编程语言,自 2015 年 5 月发布 1.0 之后,一直以每 6 周
一个小版本的开发进度稳定向前推进。语言设计上跟 C++ 一样强调零开销抽象和 RAII。
拥有极小的运行时和高效的 C 绑定,使其运行效率与 C/C++ 一个级别,非常适合对性能
要求较高的系统编程领域。利用强大的类型系统和独特的生命周期管理实现了编译期内存管理,
保证内存安全和线程安全的同时使编译后的程序运行速度极快,Rust 还提供函数式编程语言
的模式匹配和类型推导,让程序写起来更简洁优雅。宏和基于 trait 的泛型机制让 Rust
的拥有非常强大的抽象能力,在实际工程中尤其是库的编写过程中可以少写很多 boilerplate
代码。\cite{bib:2-why-rust}
总地来说,Rust是一门赋予每个人 构建可靠且高效软件能力的语言。\cite{bib:1-rust-lang}
它具有以下三个方面的优越性。

\subsubsection{高性能(Performance)}

Rust 速度惊人且内存利用率极高。由于没有运行时和垃圾回收,它能够胜任对性能要求
特别高的服务,可以在嵌入式设备上运行,还能轻松和其他语言集成。\cite{bib:1-rust-lang}

\begin{itemize}
\item 可执行文件:\\
  Rust是编译语言,这意味着程序直接转换为可执行的机器代码,因此可以将程序作为单个
  二进制文件进行部署;与 Python 和 Ruby 等解释型语言不同,无需随程序一起分发解释器,
  大量库和依赖项,这是一大优势。与解释型语言相比,Rust 程序非常快。\cite{bib:4-rust-go-cmp}

\item 对动态类型语言与静态类型的平衡\\
  动态类型语言在调试、运行时具有不确定性,而静态类型的语言减少了程序理解的开销和动态
  运行的不确定性,但并不是所有静态类型系统都是高效的。Rust使用可选类型对这种可能性进行
  编码,并且编译器要求你处理\texttt{None}这种情况。这样可以防止发生可怕的运行时错误
  (或等效语言),而可以将其提升为你在用户看到它之前解决的编译时错误。Rust的静态类型尽
  最大努力避免程序员的麻烦,同时鼓励长期的可维护性。一些静态类型的语言给程序员带来了沉重的
  负担,要求他们多次重复变量的类型,这阻碍了可读性和重构。其他静态类型的语言允许在全局进行
  数据类型推断。虽然在最初的开发过程中很方便,但是这会降低编译器在类型不再匹配时提供有用的
  错误信息的能力。Rust可以从这两种样式中学习,并要求顶层项(如函数参数和常量)具有显式类型,
  同时允许在函数体内部进行类型推断。\cite{bib:3-why-rust-pop}

\item 解决垃圾回收问题\\
  Rust可以选择将数据存储在堆栈上还是堆上,并在编译时确定何时不再需要内存并可以对其进行清理。
  这样可以有效利用内存,并实现更高性能的内存访问。Tilde是Rust在其Skylight产品中的早期生产用户,
  他发现通过使用Rust重写某些Java HTTP服务,他们能够将内存使用量从5Gb减少到50Mb。无需连续运行
  垃圾收集器,Rust项目非常适合被其他编程语言通过外部功能接口用作库。这使现有项目可以用快速的Rust
  代码替换对性能至关重要的代码,而不会产生其他系统编程语言固有的内存安全风险。某些项目甚至已使用
  这些技术在Rust中进行了增量重写。通过直接访问硬件和内存,Rust是嵌入式和裸机开发的理想语言你您
  可以编写底层代码,例如操作系统内核或微控制器应用程序。在这些特别具有挑战性的环境中,Rust的核心
  类型和功能以及可重用的库代码表现将会非常出色。\cite{bib:3-why-rust-pop}
\end{itemize}

\subsubsection{可靠性(Reliability)}

Rust 丰富的类型系统和所有权模型保证了内存安全和线程安全,让您在编译期就能够消除各种各样的错误。\cite{bib:1-rust-lang}

\begin{itemize}
\item 处理系统级编程\\
  与其他系统级编程语言(例如C或C ++)相比,Rust可以提供的最大好处是借阅检查器。
  这是编译器的一部分,负责确保引用不会超出引用的数据寿命,并有助于消除由于内存不安全而导致的
  所有类型的错误。与许多现有的系统编程语言不同,Rust不需要你将所有时间都花在细节上。Rust力求
  拥有尽可能多的\textit{零成本抽象},这种抽象与等效的手写代码具有同等的性能。当安全的Rust无法表达某些
  概念时,\url{http://cliffle.com/p/dangerust/}可以使用不安全的 Rust。这样可以释放一些
  额外的功能,但作为交换,程序员现在有责任确保代码真正安全。然后,可以将这种不安全的代码包装在
  更高级别的抽象中,以确保抽象的所有使用都是安全的。使用不安全的代码应该是一个经过深思熟虑的决定,
  因为正确使用它需要与负责避免未定义行为的任何其他语言一样多的思考和关心。最小化不安全代码是最小化
  由于内存不安全而导致段错误和漏洞的可能性的最佳方法。系统性编程语言有一种天生的使命,即它们必须
  永远有效地存在。尽管某些现代软件并不需要那么长的使用寿命,
  但许多企业希望其辛苦编写代码库在可预见的将来能够一直使用。\cite{bib:3-why-rust-pop}

\item Concurrent programming made easier\\
  Rust makes it easier to write concurrent programs by preventing data races at
  compile time. A data race occurs when at least two different instructions from
  different threads are trying to access the same memory location simultaneously,
  while at least one of them is trying to write something and there is no synchronization
  that could set any particular order among the various accesses. Access to the memory
  without synchronization is undefined. In Rust, data races are detected. If a given
  object access does not support many threads ( i. e. is not marked with an appropriate
  trait), it needs to be synchronized by a mutex that will lock access to this particular
  object for other threads. To ensure that operations performed on an object will not
  break it, only one thread has access to it. From the perspective of other threads,
  operations on this object are atomic, which means that an observed state of the
  object is always correct and you cannot observe any intermediate state resulting
  from an operation performed on this object by another thread. Rust language can
  check if we are performing any incorrect operations on such objects and inform
  us at compile time.\cite{bib:6-why-rust-pop-2}
\end{itemize}

\subsubsection{生产力(Productivity)}

Rust 拥有出色的文档、友好的编译器和清晰的错误提示信息,还集成了一流的工具——包管理器
和构建工具, 智能地自动补全和类型检验的多编辑器支持, 以及自动格式化代码等等。\cite{bib:1-rust-lang}

\begin{itemize}
\item Cargo包管理器\\
  Rust 由于有 Cargo 这样一个非常出色的包管理工具,周边的第三方库发展非常迅速,
  各个领域都有比较成熟的库,比如 HTTP 库有 Hyper,异步 IO 库有 Tokio, mio 等,
  基本上构建后端应用必须的库 Rust 都已经比较齐备。 总体来说,现阶段 Rust 定位的方向
  还是高性能服务器端程序开发,另外类型系统和语法层面上的创新也使得其可以作为开发 DSL
  的利器。\cite{bib:2-why-rust}\\
  Cargo is recognized as one of the exceptional strengths of the Rust ecosystem.
  Without Cargo, we would have had to search for libraries, download these
  libraries from unknown sources from GitHub, build as static library crates,
  link them against the program. How painful this is all to do. But we have
  Cargo that does all this for us while working with Rust.\cite{bib:7-rust-by-num}
\end{itemize}
\subsubsection{Rust语言的兼容性}

The Rust language is fast evolving, and because of this certain compatibility
issues can arise, despite efforts to ensure forwards-compatibility wherever possible.

Rust, like many programming languages, has the concept of ``keywords". These
identifiers mean something to the language, and so you cannot use them in places
like variable names, function names, and other places. Raw identifiers let you use
keywords where they would not normally be allowed. This is particularly useful
when Rust introduces new keywords, and a library using an older edition of Rust
has a variable or function with the same name as a keyword introduced in a newer edition.

For example, consider a crate \texttt{foo} compiled with the 2015 edition of Rust that
exports a function named \texttt{try}. This keyword is reserved for a new feature in the
 2018 edition, so without raw identifiers, we would have no way to name the function.\cite{bib:8-rust-compatibility}

\subsubsection{相比其他语言Rust的优势}

\begin{itemize}
\item Go:Rust语言表达能力更强,性能更高,同时线程安全方面Rust也更强,不容易写出错误的代码,
包管理Rust也更好,Go虽然在1.10版本后提供了包管理,但是目前还比不上Rust的。
\item C++:与C++相比,Rust的性能相差无几,但是在安全性方面会更优,特别是使用第三方库时,
Rust的严格要求会让第三方库的质量明显高很多。语言本身的学习,Rust的前中期学习曲线会更陡峭,
但是对于未来使用场景和生态的学习,C++会更难、更复杂。
\item Java:除了极少部分纯粹的数字计算性能,Rust的性能是全面领先于Java的,同时Rust占用内存小的多,
因此实现同等规模的服务,Rust所需的硬件成本会显著降低。
\item Python:性能自然是Rust完胜,同时Rust对运行环境要求较低,这两点差不多就足够抉择了,
因为python和rust的彼此适用面其实不太冲突。\cite{bib:5-why-rust2}
\end{itemize}

\subsection{RISC-V指令集架构}%TODO

\subsection{目前的Unikernel}
\subsubsection[ClickOS]{ClickOS\cite{bib:12-clickos}\cite{bib:13-clickos2}}\sectionauthor{蓝俊玮}


ClickOS 是一个基于 Xen 的高性能地虚拟化软件中间盒平台。为了达到高性能,
ClickOS 对 Xen 的 I/O 子系统实现了广泛的翻修,包括对后端交换机、
虚拟网络设备和后端前端驱动程序。这些更改使 ClickOS 能够显著加快中间盒运行时的网络连接。

ClickOS 虚拟机很小(5mb) ,启动速度很快(大约30毫秒) ,只需要很少的延迟(45微秒) 。
ClickOS 可以将网络功能虚拟化变成现实:它在商品硬件上运行数百个中间盒,
提供每秒数百万个数据包的处理速度并产生低数据包延迟。ClickOS 证明软件解决方案本
身就足以显着加快虚拟机处理速度,达到剩余开销的地步与将异构中间盒处理安全地整合到
同一硬件上的能力相形见绌。

ClickOS 的主要贡献是采用 Click 作为中间盒的主要编程抽象并创建量身定制的客户操作系统
运行 Click 配置。这种专业化使我们能够优化中间盒的运行时间以适应他们以毫秒为单位启动的点,
同时允许我们支持广泛的功能。ClickOS 实现了广泛的中间盒,包括防火墙、运营商级 NAT 和
负载均衡器,并证明 ClickOS 可以每秒处理数百万个数据包,达到生产级性能。

ClickOS 可以帮助测试和部署通过将流子集引导到运行实验代码的 VM 来获得新功能;
功能问题会那么只影响一小部分流量,甚至虚拟机 崩溃不会是一个大问题,因为它们可以在几毫秒内
重新实例化。

\begin{figure}[H]
\includegraphics[width=\linewidth]{pictures/clickOS_arch.jpg}
\caption{ClickOS 的架构}
\end{figure}
\begin{figure}[H]
\includegraphics[width=\linewidth]{pictures/ClickOS_networking.png}
\caption{Basic ClickOS networking in Xen}
\end{figure}

\subsubsection[MirageOS]{MirageOS\cite{bib:11-unikerel2}}\sectionauthor{蓝俊玮}

MirageOS 是一个基于 OCaml 语言,发行在 Xen hypervisor,
用于在各种云计算和移动平台构建安全、高性能网络应用程序的库操作系统。
它可以将大型服务器划分为很多更小的虚拟机,使得服务器具有更强的拓展性和安全性。
其代码可以在 Linux 、Mac OS 等系统中开发,然后编译成一个完全独立的、专门的内核
在 Xen 或者 KVM hypervisors 和 轻量级 hypervisors 下运行。MirageOS
已经发展成为一个由近100个开放源码库组成的成熟库,实现了一系列广泛的功能,
并且正开始与 Citrix XenServer 等商业产品集成。其工作原理如下,MirageOS
将 Xen hypervisors当成一个稳定的硬件平台,让我们可以专注于实施高性能协议,
没必要为支持传统操作系统里面的成千上万个设备驱动程序而操心。

MirageOS 的架构如图\ \ref{fig:mirage-fig1}。
\begin{figure}[!hbt]
\includegraphics[width=\linewidth]{pictures/figure1.png}
\caption{}\label{fig:mirage-fig1}
\end{figure}

其逻辑工作流如图\ \ref{fig:mirage-fig2}。来自源代码(本地和全局库)和
配置文件的精确依赖性跟踪使得已部署的内核二进制文件的完整出处被记录在
不可变数据存储中,足以根据需要精确地重新编译。

\begin{figure}[!hbt]
\includegraphics[width=\linewidth]{pictures/figure2.png}
\caption{}\label{fig:mirage-fig2}
\end{figure}

图\ \ref{fig:mirage-fig1}\ 说明了 MirageOS 的设计。
与传统的云部署周期相比,它赋予编译器更广泛的源代码依赖视角:

\begin{itemize}
\item 输入应用程序的所有源代码依赖项都被显式跟踪,包括实现内
核功能所需的所有库。MirageOS 包括一个构建系统,该系统内部使用
一个 SAT 解算器(使用 OPAM 包管理器,使用 Mancoosi 项目的解算器)
从一个已发布的在线包集中搜索兼容的模块实现。由于 OCaml 的静态类型
检查,在编译时将捕获接口中的任何不匹配。
\item  编译器可以输出一个完整的独立内核,而不仅仅是一个 Unix
可执行文件。这些统一内核是单用途的 libOS 虚拟机,它们只执行应用程序
源文件和配置文件中定义的任务,并且它们依赖于管理程序来提供资源复用和
隔离。甚至启动装载程序(它必须设置虚拟内存页表并初始化语言运行时)也是
作为一个简单的库编写的。每个应用程序都链接到它需要的特定库集合,并可
以以特定于应用程序的方式将它们粘合在一起。
\item 专门的单一内核部署在公共云上。与传统的虚拟化等价物相比,它们的
攻击面要小得多,并且在引导时间、二进制大小和运行时性能方面更具资源效率。
\end{itemize}

在 MirageOS 中,OCaml 编译器接收整个内核代码的源代码,
并将其链接到一个独立的本机代码对象文件。它链接到提供引导支持和
垃圾回收器的最小运行时。没有抢占式线程,内核是通过一个 I/O 循环轮询
Xen 设备的事件驱动的。

图\ \ref{fig:mirage-fig5}\ 比较了 MirageOS 和 Linux/Apache
发行版中服务的引导时间。简化的 Linux 内核和 MirageOS 的启动时间是相似的,
但一旦需要初始化用户空间应用程序,效率低下的问题就会蔓延到 Linux。
一旦启动,MirageOS unikernel 就可以提供通信服务。

\begin{figure}[!hbt]
\includegraphics[width=\linewidth]{pictures/figure5.png}
\caption{}\label{fig:mirage-fig5}
\end{figure}

\subsubsection[IncludeOS]{IncludeOS\cite{bib:9-includeos}\cite{bib:10-includeos2}}\sectionauthor{蓝俊玮}

IncludeOS 是一个为开发基于 unikernel 的应用程序而创建 C++ API 的项目。
当使用 IncludeOS 构建应用程序时,开发工具链将链接到运行应用程序所需的 IncludeOS
库的各个部分,并创建带有引导加载程序的磁盘映像。一个 IncludeOS 映像可以比运行同等
程序的 Ubuntu 系统映像小几百倍。映像的启动时间为数百毫秒,这使得快速启动许多此类
虚拟机映像成为可能。

当 IncludeOS 映像引导时,它通过设置内存、运行全局构造函数、注册驱动程序和中断处理
程序来初始化操作系统。在 IncludeOS unikernel 中,不启用虚拟内存,应用程序和
 unikernel 库使用单个地址空间。因此,没有系统调用或用户空间的概念; 所有操作系统服务
 都通过对库的简单函数调用来调用,并且都以特权模式运行。

IncludeOS 有如下优点:
\begin{itemize}
\item 性能优良,启动迅速,能在几十毫秒之内启动。
\item 可移植性好,运行在虚拟机上。
\item 安全性好,镜像中没有冗余代码。
\item 体积小,只需很小的磁盘和内存。
\item 支持在裸机上运行。
\item 延迟很低。目前没有进程抢占,操作系统的行为非常静态,
所以只要机器本身是可预测的,延迟也将是完全可预测的。因此,在裸机硬件上,
IncludeOS可被视为低延迟,可预测的操作系统。
\item 对网络的支持很好,与 Linux 相比表现出色。
\item IncludeOS 系统作为一个整体进行编译和优化。在编译器和连接器阶段,
优化器可以更多地了解整个系统正在做什么,并且有可能进一步优化。
\item IncludeOS 中的所有 IRQ 处理程序将简单地(原子地)更新计数器,
并在有时间时将进一步的处理推迟到主事件循环。这消除了对上下文切换的需要,
同时也消除了与并发相关的问题,如竞争条件。通过使所有 I/O 都是异步的,CPU 保持忙碌,这样就不会发生阻塞。
\end{itemize}

IncludeOS 有如下缺点:
\begin{itemize}
\item IncludeOS 不实现所有 POSIX。开发人员认为,
只有在需要时才会实现 POSIX 的某些部分。开发人员不太可能将完全遵守 POSIX 作为一个目标。
\item 目前,IncludeOS 中没有实现阻塞调用,因为当前的事件循环模型是使用它的最佳方式。
\item IncludeOS 目前还缺少可写的文件系统。
\end{itemize}

\subsubsection[RustyHermit]{RustyHermit\cite{bib:14-rusty-hermit}\cite{bib:15-rust-runtime}\cite{bib:16-rust-runtime2}\cite{bib:17-rusty-hermit2}}\sectionauthor{陈建绿}

RustyHermit(\hyperlink{https://github.com/hermitcore/rusty-hermit}{Github}) 是
一个基于 Rust 的、轻量级的 Unikernel。它用 Rust 语言完全改写了 RWTH Aachen University
 开发的研究项目 \hyperlink{http://hermitcore.org/}{HermitCore}。

\begin{quote}
HermitCore 最初是用 C 语言编写的,是一种针对高性能和云计算的可伸缩和可预测的运行时的 Unikernel。
\end{quote}

该项目完全使用 Rust 语言开发,\textbf{Rust 的所有权模型保证了它的内存/线程安全},
并且让开发者能够在编译时就消除许多种 bug。因此,与通用编程语言相比,使用 Rust
 进行内核开发会留下更少的漏洞,得到更加安全的内核。

开发者\textbf{扩展了 Rust 工具链}以至于 RustyHermit 的 build 过程
与 Rust 通常的工作流程相似。使用 Rust runtime 而且不直接使用 OS 服务的
 Rust 应用程序能够直接在 RustyHermit 上运行而不需要修改。因此,\textbf{原则上,
 每一个现有的 Rust 应用程序都可以建立在 RustyHermit 之上}。

Rust runtime,翻译过来称为 Rust 运行时。其中 runtime 这个词在维基百科中的定义如下:
\begin{quote}
In computer science, \textbf{runtime}, \textbf{run time}, or \textbf{execution time} is \textit{the final phase of a computer program's life cycle} in which the code is being executed on the computer's central processing unit (CPU) as machine code. In other words, ``runtime" is the running phase of a program.
\end{quote}
由此可以将 runtime 简单理解为\textbf{程序代码被 CPU 执行的那段时间}。

RustyHermit 中\textbf{优化实现了网络栈}。它使用 \hyperlink{https://github.com/smoltcp-rs/smoltcp}{smoltcp} (Rust 语言编写)
作为它的网络栈,使用 \hyperlink{https://www.linux-kvm.org/page/Virtio}{Virtio}
(KVM 的准虚拟化驱动程序,广泛应用于虚拟化 Linux 环境中) 作为客户机和主机操作系统之间的接口。
将RustyHermit 和 Linux 分别作为客户端运行在基于 Linux 的主机系统上的虚拟机中,以信息的比特数作为自变量,吞吐量/Mbps作为因变量,
进行测试并绘图,结果如下:\cite{bib:20-linux-kernel}
\begin{figure}[H]
\includegraphics[width=\linewidth]{pictures/RustyHermit-1.png}
\caption{}
\end{figure}

由结果图可以看出,\textbf{RustyHermit 在信息比特数较小时吞吐量明显比 Linux 更快}。

RustyHermit 也是一个用来评估操作系统新的设计的研究项目。
比如,RustyHermit 提供了一些经典的技术来提升像堆栈保护、应用程序堆栈
与操作系统库堆栈分离等行为的安全性。但是,库操作系统通常使用一个普通函数
调用进入内核,传统的通过进入更高的权限级别来将用户空间和内核空间分离的做法是被遗漏了的。

有Sung, Olivier, Lankes and Ravindran\cite{bib:18-intra-unikernel}
提出了\textbf{一个修改版本的 RustyHermit},该版本提供了一
个\textbf{使用 Intel MPK}(Memory Protection Keys)\textbf{进行内部隔离}的
Unikernel。这篇论文的摘要如下(中英文对照):\cite{bib:19-mpk}

\begin{quote}
Unikernels are minimal, single-purpose virtual machines. This new operating system model promises numerous benefits within many application domains in terms of lightweightness, performance, and security. Although the isolation between unikernels is generally recognized as strong, there is no isolation within a unikernel itself. This is due to the use of a single, unprotected address space, a basic principle of unikernels that provide their lightweightness and performance benefits. In this paper, we propose a new design that brings memory isolation inside a unikernel instance while keeping a single address space. We leverage Intel’s Memory Protection Key to do so without impacting the lightweightness and performance benefits of unikernels. We implement our isolation scheme within an existing unikernel written in Rust and use it to provide isolation between trusted and untrusted components: we isolate (1) safe kernel code from unsafe kernel code and (2) kernel code from user code. Evaluation shows that our system provides such isolation with very low performance overhead. Notably, the unikernel with our isolation exhibits only 0.6\% slowdown on a set of macrobenchmarks.
\end{quote}

译:

\begin{quote}
Unikernels 是最小的、单一用途的虚拟机。这种新的操作系统模型在许多应用程序领域中承诺了轻量化、性能和安全性方面的许多好处。虽然 Unikernels 之间的隔离通常被认为是很强的,但是在 Unikernels 本身内部并没有隔离。这是因为使用了单一的、不受保护的地址空间,这是提供轻量级和性能优势的 Unikernels 的基本原则。在本文中,我们提出了一种新的设计,即在保持单一地址空间的同时,在 Unikernel 实例中引入内存隔离。我们利用 Intel 的内存保护密钥(MPK)来做到这一点,而不会影响 Unikernels 的轻量化和性能优势。我们在现有的用 Rust 编写的 Unikernel\footnote{“用 Rust 编写的 Unikernel”指 RustyHermit。} 中实现了我们的隔离方案,并使用它来提供可信和不可信组件之间的隔离:(1)安全内核代码与不安全内核代码之间的隔离,(2)内核代码与用户代码之间的隔离。评估表明,我们的系统以非常低的性能开销提供了这种隔离。值得注意的是,在一组宏基准测试中,带有隔离功能的 unikernel 仅减慢了0.6\%。
\end{quote}

这篇论文中的内容可以作为我们实现 runikraft 的参考。

\subsubsection{Rumprun}\sectionauthor{陈建绿}

Rumprun unikernel 是在 rump kernels 的基础上开发的。
Rumprun 不仅可以\textbf{在像 KVM 和 Xen 这样的管理程序上工作},
还\textbf{可以在裸金属上工作}。\textbf{无论有没有 POSIX-y 接口,Rumprun 都
可以正常使用}。如果有 POSIX-y 接口,Rumprun 则允许现有的、
未经修改的 POSIX 应用程序开箱即用;如果没有 POSIX-y 接口,
Rumprun 则允许构建高度自定义的解决方案,并且占用的空间最小。

Rumprun unikernel 支持用 C、 C++ 、 Erlang、 Go、
Java、 Javascript (node.js)、 Python、 Ruby 和 Rust 等语言编写的应用程序。

在 \hyperlink{https://github.com/rumpkernel/rumprun-packages}{rumprun-packages repository}
中可以找到用于 Rumprun 的现成软件包,比如 \texttt{LevelDB},
\texttt{Memcached}, \texttt{nanomsg}, \texttt{Nginx} 和 \texttt{Redis}。

\paragraph{Rump kernels 的相关介绍}\cite{bib:21-rump-kernel}\cite{bib:22-xen}

Rump kernels 的目标并不是搭建一个 unikernel,它的目标是提供可重用的内核组件,
其他组件可以在此基础上进行构建。开发者的目标是让每个使用它们组件的项目都不用去花费精力维护轮子。

Rump kernels 的组件来自未经修改的 NetBSD,由此开发者提供了一个 POSIX-y API。
Rump Kernel 项目以一种可用于构建轻量级、特殊用途虚拟机的形式提供了 NetBSD 的
模块化驱动程序。因为开发者没有做会将错误引入到应用程序运行时(application runtime)、
libc 或驱动程序中的移植工作,所以程序可以很稳定地工作。下面这张图片阐述了 Anykernel、
Rump kernel 和 Rumprun Unikernel 的关系:
\begin{figure}[H]
\includegraphics[width=\linewidth]{pictures/rumprun-1.png}
\caption{}
\end{figure}

\begin{quote}
“Anykernel”概念指的是一种与架构无关的驱动程序方法,在这种方法中,驱动程序既可以编译到宏内核中,也可以作为用户空间进程运行,具有微内核风格,并且不需要修改代码。\\
\hfill——维基百科
\end{quote}

2. \paragraph{Rumprun 的相关介绍}\cite{bib:23-mirageos}\cite{bib:24-rumrun}

目前已经有很多 Unikernel 项目,它们的实现方式大致可以分为两种:
\begin{itemize}
\item 全新的方式(Clean-slate):在构建单一用途的操作系统的假设下,
自由地使用现代工具来进行构建,比如模块化(modularity)、
声明性代码(declarative code)、避开样板文件(avoiding boilerplate)等。并且从头开始思考操作系统和应用程序层的实现,
使用高级语言进行系统库的编写,从而使得实现更加可掌控,得到的系统库质量更高。
\item 传统的方式(Legacy):在不进行修改或只进行一些小的修改的前提下,
运行现有的软件。这通常通过将现有的操作系统代码库重构到库操作系统中来实现。
\end{itemize}

用 OCmal 语言编写的 MirageOS Unikernel 就是使用 Clean-slate 方式实现的,
而用 C 语言编写的 Rumprun Unikernel 则是使用 Legacy 方式实现的。

Rumprun 可用于将几乎任何与 POSIX 兼容的程序转换为一个可工作的
Unikernel。使用 Rumprun,理论上可以将 Linux 或者 类Unix系统上的大部分
程序编译成 Unikernel。Rumprun 以开发 NetBSD 内核中的驱动程序并在用户空间
中进行测试的需求为出发点,主要的工作是重构这个代码库,使其看起来像一个库操作系统。

下面是 Rumprun 的架构图:
\begin{figure}[H]
\includegraphics[width=\linewidth]{pictures/rumprun-2.png}
\end{figure}

Rumprun 也有一些限制:

\begin{itemize}
\item single address-space
    \begin{itemize}
    \item no processes
    \item no virtual memory
    \item no signals
    \end{itemize}
\item toolchain
    \begin{itemize}
    \item still experimental
    \end{itemize}
\item threading
    \begin{itemize}
    \item cooperative
    \item single-core
        \begin{itemize}
        \item need to spawn multiple unikernels to use multiple cores
        \end{itemize}
    \end{itemize}
\end{itemize}

下图是 Rumprun 的一个工作流程示例图:
\begin{figure}[H]
\includegraphics[width=\linewidth]{pictures/rumprun-3.png}
\caption{}
\end{figure}

我在调研的过程中发现,rump kernel 的好多官方文档都会重定向到
\url{https://rumpkernel.org}这个网址,而这个网址目前只有一些 IT News,
并非和 rump kernel 相关的内容,所以\textit{猜测该项目目前已经无人维护}。

\subsubsection{Nanos}\sectionauthor{陈建绿}

我在调研过程中发现一个比较新的正在开发中的 Unikernel:\hyperlink{https://github.com/nanovms/nanos}{Nanos(Github)}。下面是它的一些介绍:

\begin{itemize}
\item Nanos 是一个新的内核,旨在虚拟化环境中运行一个且仅有一个应用程序。
与 Windows 或 Linux 等通用操作系统相比,它有几个限制——即它是一个单进程系统,
不支持运行多个程序,也不具备通过 ssh 进行用户或远程管理的概念。
\item Nanos 的目标是成为一个比 Linux 安全得多的系统。
它做到这一点的几个依赖:\textbf{没有用户的概念},\textbf{每个虚拟机只运行一个进程},\textbf{限制每个虚拟机中包含的代码数量}。
\item Nanos 并不打算在裸金属上运行,所以开发者努力使\textbf{其内核尽可能简单}。
\end{itemize}

这也许会对我们的项目有所帮助。
\section{立项依据}

\section{前瞻性/重要性分析}

\section{相关工作}

\begin{thebibliography}{99}
\bibitem{bib:unikernel} \rule[0.5ex]{2em}{0.4pt}.
\textit{Unikernels: Rethinking Cloud Infrastructure}[Z/OL].
@unikernel [2022-02-18]. \url{https://web.archive.org/web/20220218194213/http://unikernel.org/}

\bibitem{bib:unikraft} Simon Kuenzer, Vlad-Andrei Bădoiu, Hugo Lefeuvre, Sharan Santhanam,
Alexander Jung, Gaulthier Gain, Cyril Soldani, Costin Lupu, \c{S}tefan Teodorescu, Costi Răducanu,
Cristian Banu, Laurent Mathy, Răzvan Deaconescu, Costin Raiciu and Felipe Huici.
\textit{Unikraft: Fast, Specialized Unikernels the Easy Way}[J/OL]. EuroSys '21, 2021, April: 26–29
[2022-03-27]. \url{https://dl.acm.org/doi/10.1145/3447786.3456248} \texttt{doi:10.1145/3447786.3456248}.

\bibitem{bib:unikernel-secuirty} Spencer Michaels and Jeff Dileo. \textit{Assessing Unikernel Security
}[R/OL]. Version 1.0. NCC Group, 2019: 4-10 [2022-03-27].
\url{https://research.nccgroup.com/wp-content/uploads/2020/07/ncc_group-assessing_unikernel_security.pdf}

\bibitem{bib:1-rust-lang} Rust Programming Language \url{https://www.rust-lang.org/}
\bibitem{bib:2-why-rust} 我们为什么要选择小众语言 Rust 来开发软件?\url{https://www.techug.com/post/why-we-choose-rust-to-dev.html}
\bibitem{bib:3-why-rust-pop} What is Rust and why is it so popular? - Stack Overflow Blog \url{https://stackoverflow.blog/2020/01/20/what-is-rust-and-why-is-it-so-popular/}
\bibitem{bib:4-rust-go-cmp} 也许是最客观、全面的比较 Rust 与 Go:都想把 Rust 也学一下\url{https://www.cnblogs.com/Chary/p/14097609.html}
\bibitem{bib:5-why-rust2} 为什么要使用 Rust 语言?Rust 语言的优势在哪里?\url{https://www.zhihu.com/question/393796866}
\bibitem{bib:6-why-rust-pop-2} [Rust programming language - what is rust used for and why is so popular?]\url{https://codilime.com/blog/why-is-rust-programming-language-so-popular/}
\bibitem{bib:7-rust-by-num} [Rust by the Numbers: The Rust Programming Language in 2021 – The New Stack]\url{https://thenewstack.io/rust-by-the-numbers-the-rust-programming-language-in-2021/}
\bibitem{bib:8-rust-compatibility} [Compatibility - Rust By Example (rust-lang.org)]\url{https://doc.rust-lang.org/rust-by-example/compatibility.html}
\bibitem{bib:9-includeos}  [IncludeOS: A minimal, resource efficient unikernel for cloud systems]\url{https://blog.acolyer.org/2016/02/22/includeos}
\bibitem{bib:10-includeos2} [IncludeOS: a unikernel for C++ applications]\url{https://lwn.net/Articles/728682/}
\bibitem{bib:11-unikerel2} [Unikernels: Rise of the Virtual Library Operating System]\url{https://queue.acm.org/detail.cfm?id=2566628}
\bibitem{bib:12-clickos} [ClickOS and the Art of Network Function Virtualization]\url{http://cnp.neclab.eu/projects/clickos/clickos.pdf}
\bibitem{bib:13-clickos2} [Enabling Fast, Dynamic Network Processing with ClickOS]\url{http://cnp.neclab.eu/projects/clickos/clickos-workshop.pdf}
\bibitem{bib:14-rusty-hermit} [rusty-hermit 0.3.10 doc]\url{https://docs.rs/crate/rusty-hermit/0.3.10}
\bibitem{bib:15-rust-runtime} [Rust Runtime 与 ABI——知乎专栏]\url{https://zhuanlan.zhihu.com/p/370897059}
\bibitem{bib:16-rust-runtime2} [Rust 运行时 - Rust 参考 (rust-lang.org)]\url{https://doc.rust-lang.org/reference/runtime.html}
\bibitem{bib:17-rusty-hermit2} [The `RustyHermit` Unikernel——Rust OSDev]\url{https://rust-osdev.com/showcase/rusty-hermit/}
\bibitem{bib:18-intra-unikernel} [Intra-Unikernel Isolation with Intel Memory Protection Keys.pdf]\url{../../references/Intra-Unikernel Isolation with Intel Memory Protection Keys.pdf}
\bibitem{bib:19-mpk} [MPK——Core API Doc]\url{https://www.kernel.org/doc/html/latest/core-api/protection-keys.html#:~:text=Memory%20Protection%20Keys%20provides%20a%20mechanism%20for%20enforcing,to%20a%20%E2%80%9Cprotection%20key%E2%80%9D%2C%20giving%2016%20possible%20keys.}
\bibitem{bib:20-linux-kernel} [linux内核那些事之Memory protection keys(硬件原理)——CSDN博客]\url{https://blog.csdn.net/weixin_42730667/article/details/121386896}
\bibitem{bib:21-rump-kernel} [Rump kernel——Wikipedia (其中有介绍 Anykernel)]\url{https://en.wikipedia.org/wiki/Rump_kernel}
\bibitem{bib:22-xen} [Xen on Rump Kernels and the Rumprun Unikernel——XenProject]\url{https://xenproject.org/2015/08/06/on-rump-kernels-and-the-rumprun-unikernel/}
\bibitem{bib:23-mirageos} [All About Unikernels: Part 2, Two Different Approaches, MirageOS and Rumprun——Container Solutions blog]\url{https://blog.container-solutions.com/all-about-unikernels-part-2-mirageos-and-rumprun}
\bibitem{bib:24-rumrun} [The Rumprun Unikernel]\url{../../references/The Rumprun Unikernel.pdf}
\end{thebibliography}
\end{document}